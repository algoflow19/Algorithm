%
% 6.006 problem set 6
%
\documentclass[12pt,twoside]{article}

\input{macros}

\usepackage{amsmath}
\usepackage{url}
\usepackage{mdwlist}
\usepackage{graphicx}
\usepackage{clrscode3e}
\newcommand{\isnotequal}{\mathrel{\scalebox{0.8}[1]{!}\hspace*{1pt}\scalebox{0.8}[1]{=}}}
\usepackage{listings}
\usepackage{tikz}
\usetikzlibrary{arrows}
\usetikzlibrary{matrix}
\usetikzlibrary{positioning}
\usetikzlibrary{shapes.geometric}
\usetikzlibrary{shapes.misc}
\usetikzlibrary{trees}

\newcommand{\answer}{
 \par\medskip
 \textbf{Answer:}
}

\newcommand{\collaborators}{ \textbf{Collaborators:}
%%% COLLABORATORS START %%%
None.
%%% COLLABORATORS END %%%
}

\newcommand{\answerIa}{ \answer
%%% PROBLEM 1(a) ANSWER START %%%
3
%%% PROBLEM 1(a) ANSWER END %%%
}

\newcommand{\answerIb}{ \answer
%%% PROBLEM 1(b) ANSWER START %%%
3
%%% PROBLEM 1(b) ANSWER END %%%
}

\newcommand{\answerIc}{ \answer
%%% PROBLEM 1(c) ANSWER START %%%
1
%%% PROBLEM 1(c) ANSWER END %%%
}

\newcommand{\answerId}{ \answer
%%% PROBLEM 1(d) ANSWER START %%%
4
%%% PROBLEM 1(d) ANSWER END %%%
}
\newcommand{\answerIe}{ \answer
%%% PROBLEM 1(e) ANSWER START %%%
3
%%% PROBLEM 1(e) ANSWER END %%%
}
\newcommand{\answerIf}{ \answer
%%% PROBLEM 1(f) ANSWER START %%%
2
%%% PROBLEM 1(f) ANSWER END %%%
}
\newcommand{\answerIg}{ \answer
%%% PROBLEM 1(g) ANSWER START %%%
3
%%% PROBLEM 1(g) ANSWER END %%%
}
\newcommand{\answerIh}{ \answer
%%% PROBLEM 1(h) ANSWER START %%%
3
%%% PROBLEM 1(h) ANSWER END %%%
}
\newcommand{\answerIi}{ \answer
%%% PROBLEM 1(i) ANSWER START %%%
4
%%% PROBLEM 1(i) ANSWER END %%%
}
\newcommand{\answerIj}{ \answer
%%% PROBLEM 1(j) ANSWER START %%%
2
%%% PROBLEM 1(j) ANSWER END %%%
}
\newcommand{\answerIk}{ \answer
%%% PROBLEM 1(k) ANSWER START %%%
4
%%% PROBLEM 1(k) ANSWER END %%%
}
\newcommand{\answerIl}{ \answer
%%% PROBLEM 1(l) ANSWER START %%%
8
%%% PROBLEM 1(l) ANSWER END %%%
}
\newcommand{\answerIm}{ \answer
%%% PROBLEM 1(m) ANSWER START %%%
6
%%% PROBLEM 1(m) ANSWER END %%%
}
\newcommand{\answerIn}{ \answer
%%% PROBLEM 1(n) ANSWER START %%%
6
%%% PROBLEM 1(n) ANSWER END %%%
}
\newcommand{\answerIo}{ \answer
%%% PROBLEM 1(o) ANSWER START %%%
7
%%% PROBLEM 1(o) ANSWER END %%%
}
\newcommand{\answerIp}{ \answer
%%% PROBLEM 1(p) ANSWER START %%%
\begin{codebox}
\Procname{$\proc{KTHROOT}(A,K,N)$}
\li $Size \gets (N+K-1) \div K$
\li $Result \gets \proc{Zero}(Size)$
\li \For $i \gets Size-1$ \Downto $0$ \label{li:exp-for1}
\li 	\Then $Result[i] \gets 1$
\li		$Product \gets 1$
\li 	$tmp \gets Result$
\li     \For $j \gets 0$ \To $N-1$ \label{li:exp-for2} \Then
\li        \If $K[j]==1$ \Then
\li 			$Product = Product \times tmp$ \End
\li			$tmp \gets tmp \times tmp$     \End

		\label{li:exp-mulmod2}
\li        \If $\proc{CMP}(Product,A,N) == GREATER$ 
\li 		\Then $Result[i] \gets 0$ \End \End
          \End \label{li:exp-mulmod1}       
      \End
\li \Return $Result$
\end{codebox}
\quad This sloution is similar to the official solution but not quite. Using line 1 and line 2, we first figure out the possible bit size of the result, Then we just detemind if a bit in the result is '1', if the result is too bigger by line 7 to line 11, if it is, then the bit should be '0'(line 12), we continue to detemind each bit, and finally we get the correct answer.
There are two loop which cause the inner loop code(line 7~10) could run $\Theta(n^2)$ times, since there are all times operation, we know it provide a $\Theta(n^{\log_2 3})$ complexity, so we know the algothrim run in $\Theta(N^{2 + \log_2 3})$.
%%% PROBLEM 1(p) ANSWER END %%%
}

\newcommand{\answerIIa}{ \answer
%%% PROBLEM 2(a) ANSWER START %%%
7
%%% PROBLEM 2(a) ANSWER END %%%
}
\newcommand{\answerIIb}{ \answer
%%% PROBLEM 2(b) ANSWER START %%%
3
%%% PROBLEM 2(b) ANSWER END %%%
}
\newcommand{\answerIIc}{ \answer
%%% PROBLEM 2(c) ANSWER START %%%
9
%%% PROBLEM 2(c) ANSWER END %%%
}
\newcommand{\answerIId}{ \answer
%%% PROBLEM 2(d) ANSWER START %%%
1 2 6
%%% PROBLEM 2(d) ANSWER END %%%
}
\newcommand{\answerIIe}{ \answer
%%% PROBLEM 2(e) ANSWER START %%%
1 4 5
%%% PROBLEM 2(e) ANSWER END %%%
}
\newcommand{\answerIIf}{ \answer
%%% PROBLEM 2(f) ANSWER START %%%
2 3
%%% PROBLEM 2(f) ANSWER END %%%
}

\newcommand{\answerIIIa}{ \answer 
%%% PROBLEM 3(a) ANSWER START %%%
fast\_mul
%%% PROBLEM 3(a) ANSWER END %%%
}
\newcommand{\answerIIIb}{ \answer
%%% PROBLEM 3(b) ANSWER START %%%
93496
%%% PROBLEM 3(b) ANSWER END %%%
}
\newcommand{\answerIIIc}{ \answer
%%% PROBLEM 3(c) ANSWER START %%%
4
%%% PROBLEM 3(c) ANSWER END %%%
}
\newcommand{\answerIIId}{ \answer
%%% PROBLEM 3(d) ANSWER START %%%
4
%%% PROBLEM 3(d) ANSWER END %%%
}
\newcommand{\answerIIIe}{ \answer
%%% PROBLEM 3(e) ANSWER START %%%
1 2 3
%%% PROBLEM 3(e) ANSWER END %%%
}
\newcommand{\answerIIIf}{ \answer
%%% PROBLEM 3(f) ANSWER START %%%
1
%%% PROBLEM 3(f) ANSWER END %%%
}

\setlength{\oddsidemargin}{0pt}
\setlength{\evensidemargin}{0pt}
\setlength{\textwidth}{6.5in}
\setlength{\topmargin}{0in}
\setlength{\textheight}{8.5in}

% Fill these in!
\newcommand{\theproblemsetnum}{6}
\newcommand{\releasedate}{October 24, 2011}
\newcommand{\partaduedate}{Monday, October 31}
\newcommand{\tabUnit}{3ex}
\newcommand{\tabT}{\hspace*{\tabUnit}}

\begin{document}

\handout{Problem Set \theproblemsetnum}{\releasedate}

\textbf{Both theory and programming questions} are due {\bf \partaduedate} at
{\bf 11:59PM}.
%
Please download the .zip archive for this problem set, and refer to the
\texttt{README.txt} file for instructions on preparing your solutions.

We will provide the solutions to the problem set 10 hours after the problem set
is due. You will have to read the solutions, and write a brief \textbf{grading
explanation} to help your grader understand your write-up. You will need to
submit the grading explanation by \textbf{Thursday, November 3rd, 11:59PM}. Your
grade will be based on both your solutions and the grading explanation.

\medskip

\hrulefill

\begin{problems}

\problem \points{30} \textbf{I Can Haz Moar Frendz?}

Answer:

\quad We preform a variety of breadth first search. We start the search at point S, and we stop after k steps. The mainly different things here is that we do not prevent to visit a node. we start from s, in the first step, we update all neighbors to \proc{ER}($u_0$,$u_i$) where ($u_0$,$u_i$) $\in E$ and next update each two-unit far node to \proc{ER}($u_0$,$u_j$) $ \times $ \proc{ER}($u_j$,$u_i$), if \proc{ER}($u_0$,$u_j$) $ \times $ \proc{ER}($u_j$,$u_i$) $>$ \proc{ER}($u_0$,$u_i$) or $u_i$ hasn't been visited, then repeat the process. Finally after k steps, we will get the right answer.
\newpage

\problem \points{18} \textbf{RSA Public-Key Encryption}

Answer:



\newpage

\problem \points{42} \textbf{Image Decryption}

Your manager wants to show off the power of the Knight's Shield chip by
decrypting a live video stream directly using the RSA public-key crypto-system.
RSA is quite resource-intensive, so most systems only use it to encrypt the key
of a faster algorithm. Decrypting live video would be an impressive technical
feat!

Unfortunately, the performance of the KS chip on RSA decryption doesn't come
even close to what's needed for streaming video. The hardware engineers said the
chip definitely has enough computing power, and blamed the problem on the
RSA implementation. Your new manager has heard about your algorithmic chops, and
has high hopes that you'll get the project back on track. The software engineers
suggested that you benchmark the software using images because, after all, video
is just a sequence of frames.

The code is in the \texttt{rsa} directory in the zip file for this problem set.

\begin{problemparts}
\problempart \points{2} Run the code under the python profiler with the command
below, and identify the method inside \texttt{bignum.py} that is most suitable
for optimization. Look at the methods that take up the most CPU time, and
choose the first method whose running time isn't proportional to the size of its
output.

\texttt{python -m cProfile -s time rsa.py < tests/1verdict\_32.in}

\textit{Warning:} the command above can take 1-10 minutes to complete, and
bring the CPU usage to 100\% on one of your cores. Plan accordingly. If
you have installed PyPy successfully, you should replace \texttt{python} with
\texttt{pypy} in the command above for a 2-10x speed improvement. 

What is the name of the method with the highest CPU usage?
\answerIIIa

\problempart \points{1} How many times is the method called?
\answerIIIb

\problempart \points{1} The troublesome method is implementing a familiar
arithmetic operation. What is the tightest asymptotic bound for the worst-case
running time of the method that contains the bottleneck? Express your answer in
terms of $N$, the number of digits in the input numbers.
\begin{enumerate}
  \item $\Theta(N)$.
  \item $\Theta(N \log n)$
  \item $\Theta(N \log^2 n)$
  \item $\Theta(N^{\log_{2} 3})$
  \item $\Theta(N^2)$
  \item $\Theta(N^{\log_{2} 7})$
  \item $\Theta(N^3)$
\end{enumerate}
\answerIIIc

\problempart \points{1} What is the tightest asymptotic bound for the worst-case
running time of division? Express your answer in terms of $N$, the number of
digits in the input numbers.
\begin{enumerate}
  \item $\Theta(N)$.
  \item $\Theta(N \log n)$
  \item $\Theta(N \log^2 n)$
  \item $\Theta(N^{\log_{2} 3})$
  \item $\Theta(N^2)$
  \item $\Theta(N^{\log_{2} 7})$
  \item $\Theta(N^3)$
\end{enumerate}
\answerIIId

\end{problemparts}

We have implemented a visualizer for your image decryption output, to help you
debug your code. The visualizer will also come in handy for answering the
question below. To use the visualizer, first produce a trace.

\texttt{TRACE=jsonp python rsa.py < tests/1verdict\_32.in > trace.jsonp}

On Windows, use the following command instead.

\texttt{rsa\_jsonp.bat < tests/1verdict\_32.in > trace.jsonp}

Then use Google Chrome to open
\texttt{visualizer/bin/visualizer.html}

\begin{problemparts}
\problempart \points{6} The test cases that we supply highlight the problems of
RSA that we discussed above. Which of the following is true? (True / False)
\begin{enumerate}
  \item Test \texttt{1verdict\_32} shows that RSA has fixed points.
  \item Test \texttt{1verdict\_32} shows that RSA is deterministic.
  \item Test \texttt{2logo\_32} shows that RSA has fixed points.
  \item Test \texttt{2logo\_32} shows that RSA is deterministic.
  \item Test \texttt{5future\_1024} shows that RSA has fixed points.
  \item Test \texttt{5future\_1024} shows that RSA is deterministic.
\end{enumerate}
\answerIIIe

\problempart \points{1} Read the code in \texttt{rsa.py}. Given a decrypted
image of $R \times C$ pixels ($R$ rows, $C$ columns), where all the pixels
are white (all the image data bytes are 255), how many times will
\texttt{powmod} be called during the decryption operation in
\texttt{decrypt\_image}?
\begin{enumerate}
  \item $\Theta(1)$
  \item $\Theta(R C)$
  \item $\Theta(\frac{RC}{N})$
  \item $\Theta(\frac{RN}{C})$
  \item $\Theta(\frac{CN}{R})$
\end{enumerate}
\answerIIIf

\problempart \points{30} The multiplication and division operations in
\texttt{big\_num.py} are implemented using asymptotically efficient algorithms
that we have discussed in class. However, the sizes of the numbers involved in
RSA for typical key sizes aren't suitable for complex algorithms with high
constant factors. Add new methods to \texttt{BigNum} implementing multiplication
and division using straight-forward algorithms with low constant factors, and
modify the main multiplication and division methods to use the simple algorithms
if at least one of the inputs has 64 digits (bytes) or less. Please note that
you are not allowed to import any additional Python libraries and our test will
check this.
\end{problemparts}

The KS software testing team has put together a few tests to help you check your
code's correctness and speed. \texttt{big\_num\_test.py} contains unit tests
with small inputs for all \texttt{BigNum} public methods.
\texttt{rsa\_test.py} runs the image decryption code on the test cases in the
\texttt{tests/} directory.

You can use the following command to run all the image decryption tests.

\texttt{python rsa\_test.py}

To work on a single test case, run the simulator on the test case with the
following command.

\texttt{python rsa.py < tests/1verdict\_32.in > out}

Then compare your output with the correct output for the test case.

\texttt{diff out tests/1verdict\_32.gold}

For Windows, use \texttt{fc} to compare files.

\texttt{fc out tests/1verdict\_32.gold}

While debugging your code, you should open a new Terminal window (Command Prompt
in Windows), and set the \texttt{KS\_DEBUG} environment variable (\texttt{export
KS\_DEBUG=true}; on Windows, use \texttt{set KS\_DEBUG=true}) to use a slower
version of our code that has more consistency checks.

When your code passes all tests, and runs reasonably fast (the tests should
complete in less than 90 seconds on any reasonably recent computer using PyPy,
or less than 600 seconds when using CPython), upload your modified
\texttt{big\_num.py} to the course submission site. Our automated grading code
will use our versions of \texttt{test\_rsa.py}, \texttt{rsa.py} and
\texttt{ks\_primitives.py} / \texttt{ks\_primitives\_unchecked.py}, so please do
not modify these files.

\end{problems}
\end{document}
